\documentclass[12pt, a4paper]{article}
\usepackage{ctex}

\usepackage[margin=1in]{geometry}
\usepackage{color}
\usepackage{clrscode}
\usepackage{amssymb}
\usepackage{amsthm}
\usepackage{amsmath}
\definecolor{bgGray}{RGB}{36, 36, 36}
\usepackage{supertabular}
\usepackage[
  colorlinks,
  linkcolor=bgGray,
  anchorcolor=blue,
  citecolor=green
]{hyperref}
\usepackage{listings}
\usepackage{fontspec}
\newfontfamily\courier{Courier}
\usepackage{xcolor}

\newcommand{\ilc}{\texttt}

\title{OOP大作业-Matrix}
\author{负责助教:冯思远\\场外助教组:任云玮}
\date{}

\begin{document}
\lstset{numbers=left,
  basicstyle=\scriptsize\courier,
  numberstyle=\tiny\courier\color{red!89!green!36!blue!36},
  language=C++,
  breaklines=true,
  keywordstyle=\color{blue!70},commentstyle=\color{red!50!green!50!blue!50},
  morekeywords={},
  stringstyle=\color{purple},
  frame=shadowbox,
  rulesepcolor=\color{red!20!green!20!blue!20}
}
\maketitle
\tableofcontents
\newpage

\section{概要}
  这是课程[程序设计 2017]的OOP大作业,请同学们按照文档中的要求,
  完成文件\ilc{Matrix.hpp}中的内容并回答相关问题(粗体内容)。
  
\section{实现要求}
  \begin{enumerate}
  	\item A班:手动维护一维数据结构,以行优先方式实现二维表示,禁止使用vector、deque等\textbf{任何STL容器},需通过正确性测试和鲁棒性测试。
  	\item B班:无任何限制,仅需通过正确性测试。
  	\item \textbf{强烈建议但不强制}B班有基础同学完成所有或部分A班要求。
  \end{enumerate}

\section{接口}

\subsection{构造函数与赋值}
  %回答问题:\textbf{为什么不把矩阵的大小作为模板参数?明明这样做的话,虽然
  %不能\ilc{resize},但可以在编译期就检查到诸如不合法乘法的错误。}
  \begin{enumerate}
    \item 默认构造函数;
    \item \ilc{Matrix(std::size\_t n, std::size\_t m, T init = T())},
    构造一个大小为$n\times m$的矩阵,并将里面的每个元素初始化为\ilc{init}
    \item \ilc{Matrix(std::initializer\_list<std::initializer\_list<T>>)},
    利用给定的\\\ilc{initializer\_list}来构造一个矩阵。
    \item “拷贝”构造函数与赋值。要求如果可以用一个\ilc{T}来构造一个\ilc{V},
    则可以使用一个\ilc{Matrix<T>}来构造一个\ilc{Matrix<V>}。
    \item (仅A班)“移动”构造函数与赋值。仅要求相同类型的移动构造。\textbf{思考:为什么不能跨类型移动}
  \end{enumerate}

\subsection{元素获取}
  完成如下两个函数(请务必写对),它们用于获取矩阵的第$i$行$j$列的元素,
  $0$-based。并说明\textbf{为何需要下述两个重载}
  \begin{enumerate}
    \item \ilc{T\& operator()(std::size\_t i, std::size\_t j)}
    \item \ilc{const T\& operator()(std::size\_t, std::size\_t) const}
  \end{enumerate}
  并完成如下两个函数,返回矩阵的第$i$行/列。
  \begin{enumerate}
    \item \ilc{Matrix<T> row(std::size\_t i) const}
    \item \ilc{Matrix<T> column(std::size\_t i) const}
  \end{enumerate}

\subsection{运算}
  \subsubsection{一元运算}
    \begin{enumerate}
      \item \ilc{-},对矩阵中每个元素取负。
      \item \ilc{-=}
      \item \ilc{+=}
      \item \ilc{*=},数乘
      \item \ilc{Matrix tran() const},返回当前矩阵的转置矩阵。
      注:不改变当前矩阵。
    \end{enumerate}
    要求对于\ilc{+=}等运算,支持\ilc{Matrix<int>+=}一个\ilc{Matrix<double>}

  \subsubsection{二元运算}
  \begin{enumerate}
    \item \ilc{==},比较两个矩阵是否相同(元素的值相等即可)。
    \item \ilc{!=},比较两个矩阵是否不同。
    \item \ilc{+},定义为友元函数
    \item \ilc{-},定义为友元函数
    \item \ilc{*},数乘,定义为友元函数
    \item \ilc{*},矩阵乘法,定义为友元函数
  \end{enumerate}
  对于\ilc{Matrix<T> a}和\ilc{Matrix<V> b},若\ilc{T + V}的类型为\ilc{U},
  则\ilc{a+b}的返回值类型为\ilc{Matrix<U>}(提示:\ilc{decltype})。回答问题:\textbf{为何要定义为友元函数
  而非成员函数}

\subsection{迭代器}
  要求实现一个random\_access的迭代器(具体要求见源代码),以及如下函数。
  \begin{enumerate}
    \item \ilc{iterator begin()}
    \item \ilc{iterator end()}
    \item \ilc{T\& operator*() const},返回当前迭代器指向的数据。
    \item \ilc{T* operator->() const},返回迭代器的指针。
    \item \ilc{subMatrix(std::pair<std::size\_t, std::size\_t> l,
              std::pair<std::size\_t, std::size\_t> r)}.
    设$M$是当前矩阵的一个子矩阵,\ilc{l}和\ilc{r}分别为它的左上角
    和右下角在原矩阵中的位置,返回$M$的\ilc{begin()}和\ilc{end()}
  \end{enumerate}

\subsection{其他}
  \begin{enumerate}
    \item \ilc{void clear()},清空当前矩阵,并释放内存空间。
    \item \ilc{std::size\_t rowlength()},返回行数。
    \item \ilc{std::size\_t columnlength()},返回列数。
    \item \ilc{std::pair<std::size\_t, std::size\_t> size() const},返回\ilc{size}。
    \item (仅A班)\ilc{void resize(std::size\_t n, std::size\_t m, T \_init = T())}.保留前$n*m$个元素,若元素不足则拿\ilc{\_init}补充,并重新以行优先方式组成新矩阵。若元素个数相同,则不允许重新开设内存空间。
    \item (仅A班)\ilc{void resize(std::pair<std::size\_t, std::size\_t> sz, T \_init = T())}.要求同上。
  \end{enumerate}

\section{说明}
  \begin{enumerate}
  	\item 整个项目基于C++14标准完成,不清楚如何设置的同学可以咨询助教。
  	\item 编译命令:g++ test.cpp -o test -std=C++14 -O2 -Wall -lopenblas
  	\item 尽可能项目中少出现Warning,有助于减少不必要的bug。
  	\item A、B班的评分分别进行,评分标准以对应班级助教组为准。
  \end{enumerate}

\newpage
\section{PolicyIterator(\sout{选做}不做)}

\subsection{说明}
  本节内容为OOP1作业的选做部分,内容为基于Policy-Based
  Class Design来设计三类迭代器:RowIterator, ColumnIterator
  , TraceIterator. 分别用于按行遍历,按列遍历以及遍历对角线。
  请参考后续章节中的简介或者参考书目\textit{Modern C++ Design},
  Andrei Alexandrescu.
  \par 此项内容\sout{选做}不做,\textbf{没有}bonus.

\subsection{Policy-Based Class Design}
  此节为Policy-Based Class Design的一个简介。\par
  基于Policy的设计,是为了解决如下的情况。对于库的设计者,通常
  会需要在各个方面进行权衡取舍,诸如性能和安全性,是否多线程安全
  等。为了让库的使用者可以根据需求做出选择,一种解决方法是提供所
  有的可能用到的类。诸如\ilc{ThreadSafePtr, NaivePtr, FastPtr}
  等等。但是当某一个类的可供选择的特性较多时,不同的选择之间有不
  同的组合,从而就产生了组合爆炸的问题,如果要提供所有的接口,一
  方面会产生一定的重复代码,另一方面生产的程序的大小也会相应地膨胀。
  而基于Policy的设计则是为了解决这样情况。\par
  基于Policy的设计的宗旨在于,将这些可供选择的特性包装在一些被称为
  Policy Class的类中,而用户在使用的时候,可以通过选择不同的policy
  来组合出所需要的类。在实现上,通常采用template来完成。具体可见
  如下摘录自\textit{Modern C++ Design}中的例子。
  \begin{lstlisting}
    template <class T>
    struct OpNewCreator {
        static T* Create() {
            return new T;
        }
    };

    template <class T>
    struct MallocCreator {
        static T* Create() {
            void* buf = std::malloc(sizeof(T));
            if (!buf)
                return nullptr;
            return new(buf) T;
        }
    };

    template <template<class Created> CreationPolicy>
    class WidgetManager : public CreationPolicy<Widget> {
        ...
    };

    // Application Code
    typedef WidgetManager<OpNewCreator> MyWidgetMgr;

  \end{lstlisting}

\subsection{要求}
  完成迭代器\ilc{PolicyIterator<Policy>}以及相应的Policy:
  \ilc{RowIterator, ColumnIterator, TraceIterator}.以及
  相应的\ilc{begin}和\ilc{end}. 具体内容看一眼测试就好啦OvO


\newpage
\section{参考资料}
  大家有什么问题,可以先查一下这些网站:
  \begin{enumerate}
    \item \url{en.cppreference.com/w/}
    \item \url{www.cplusplus.com}
    \item \url{stackoverflow.com}
  \end{enumerate}
  另外关于迭代器,可以参考候捷所著的《STL源码剖析》。
  \\关于C++11/14的内容,
  可以参考Scott Meyers所著的《Effective Modern C++》以及之前提到的
  网站。

\end{document}
